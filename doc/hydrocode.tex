\documentclass[a4paper]{article}
% generated by Docutils <http://docutils.sourceforge.net/>
\usepackage{fixltx2e} % LaTeX patches, \textsubscript
\usepackage{cmap} % fix search and cut-and-paste in Acrobat
\usepackage{ifthen}
\usepackage[T1]{fontenc}
\usepackage[utf8]{inputenc}
\usepackage{amsmath}
\setcounter{secnumdepth}{0}
\usepackage{tabularx}

%%% Custom LaTeX preamble
% PDF Standard Fonts
\usepackage{mathptmx} % Times
\usepackage[scaled=.90]{helvet}
\usepackage{courier}

%%% User specified packages and stylesheets

%%% Fallback definitions for Docutils-specific commands

% providelength (provide a length variable and set default, if it is new)
\providecommand*{\DUprovidelength}[2]{
  \ifthenelse{\isundefined{#1}}{\newlength{#1}\setlength{#1}{#2}}{}
}

% docinfo (width of docinfo table)
\DUprovidelength{\DUdocinfowidth}{0.9\textwidth}

% hyperlinks:
\ifthenelse{\isundefined{\hypersetup}}{
  \usepackage[colorlinks=true,linkcolor=blue,urlcolor=blue]{hyperref}
  \urlstyle{same} % normal text font (alternatives: tt, rm, sf)
}{}
\hypersetup{
  pdftitle={HydroCode documentation},
  pdfauthor={Joakim Kjellsson, Department of Meterology \& Bolin Centre for Climate Research, Stockholm University}
}

%%% Title Data
\title{\phantomsection%
  HydroCode documentation%
  \label{hydrocode-documentation}%
  \\ % subtitle%
  \large{A Fortran 95 code for calculating atmospheric stream functions}%
  \label{a-fortran-95-code-for-calculating-atmospheric-stream-functions}}
\author{}
\date{}

%%% Body
\begin{document}
\maketitle

% Docinfo
\begin{center}
\begin{tabularx}{\DUdocinfowidth}{lX}
\textbf{Author}: &
	Joakim Kjellsson, Department of Meterology \& Bolin Centre for Climate Research, Stockholm University \\
\textbf{Contact}: &
	\href{mailto:joakimkjellsson@gmail.com}{joakimkjellsson@gmail.com} \\
\end{tabularx}
\end{center}


\section{Compiling and running%
  \label{compiling-and-running}%
}
%
\begin{quote}{\ttfamily \raggedright \noindent
make\\
./psi.x
}
\end{quote}

Any dataset with data on terrain-following coordinates can be used.
Some models use sigma, $p_k = \sigma_k * p_s$, and some use hybrid, $p_k = a_k * p_0 + b_k * p_s$, or $p_k = a_k + b_k * p_s$.

The following data has been used
%
\begin{itemize}

\item %
\begin{description}
\item[{Reanalysis}] \leavevmode %
\begin{itemize}

\item ERA-Interim

\item MERRA

\end{itemize}

\end{description}

\item %
\begin{description}
\item[{CMIP5 models (atmosphere component only)}] \leavevmode %
\begin{itemize}

\item CanESM2

\item CCSM4

\item CSIRO-Mk3-6-0

\item GFDL CM3

\item IPSL CM5A

\item NorESM1

\end{itemize}

\end{description}

\end{itemize}


\section{History%
  \label{history}%
}

Parts of the original code was written by Kristofer Doos to calculate the thermohaline stream function for the NEMO ocean model.
It was then adapted for the atmosphere (ERA-Interim) by Joakim Kjellsson.
%
\begin{itemize}

\item January 2012: Original atmospheric code for ERA-Interim

\item February 2013: Added OpenMP parallellisation and GRIB->netCDF conversion with CDO

\item June 2013: Added options to use time-mean and/or zonal-mean variables. Also support for EC-Earth GFDL CM3.

\item July 2013: Restructured code into different modules, subroutines, etc. Unified different versions.

\item August 2013: Added preprocessing flags to control what stream functions are outputed.

\item September 2013: Added support from CanESM2, CCSM4, IPSL-CM5A, NorESM1

\item December 2013: Added support for CSIRO-Mk-3-6-0

\end{itemize}

\end{document}
